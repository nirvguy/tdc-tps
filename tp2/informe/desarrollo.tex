\subsection{Traceroute}

En este trabajo se desarrolló una herramienta conocida como traceroute. Esta
consta en descubir cual es la ruta a un determinado host en la transmisión 
de un paquete de una maquina a otra. En otras palabras, qué routers estan
involucrados desde que el paquete sale hasta que llega al host destinatario.

Para llevar esto a cabo hay, al menos, dos posibles alternativas. El primer
algoritmo consiste en la técnica llamada \emph{TTL seeder} sobre el protocolo
ICMP que se explicará a continuación. La segunda, especificada en el \emph{RFC
1393}, hace uso de ciertas opciones avanzadas de los datagramas IP en donde
cada router (gateway) pone su dirección IP en el paquete TCP/IP. Las ventas de este último
algoritmo es que, como se explicará mas adelante, este da información 100\%
confiable de la ruta cuando el primero no. Pero cuenta con la desventaja
de que es no es un standard obligatorio y por tanto es posible que haya
routers (gateways) que no lo implementen. Es por esto que en este trabajo se implementará
el primer algoritmo.

\subsubsection{Protocolo \textbf{ICMP}}

\textbf{ICMP} (\emph{Internet Control Message Protocol}) es un protocolo que
forma parte de la arquitectura \textbf{TCP/IP} pensado para mensajes de control
de Internet. Se utiliza en situaciones, y distintos própositos como

\begin{itemize}
	\item Cuando un datagrama no puede alcanzar su destino
	\item Cuando un router no tiene la capacidad de forwardear este paquete
	\item Redirigir el host hacia una ruta más corta para el envio del paquete
	\item Se quiere comprobar que una maquína esta viva (\textbf{ping})
	\item Se quiere saber la ruta a un determinado host (\textbf{traceroute})
\end{itemize}
