En este trabajo se desarrolló una herramienta conocida como traceroute. Esta
consta en descubir cual es la ruta a un determinado host en la transmisión 
de un paquete de un dispositivo a otro. En otras palabras, qué routers estan
involucrados desde que el paquete sale hasta que llega al host destinatario.

Para llevar esto a cabo hay, al menos, dos posibles alternativas. El primer
algoritmo consiste en la técnica llamada \emph{TTL seeder} sobre el protocolo
ICMP que se explicará a continuación. La segunda, especificada en el \emph{RFC
1393} \cite{rfc1393}, hace uso de ciertas opciones avanzadas de los datagramas IP en donde
cada router (gateway) pone su dirección IP en el paquete TCP/IP. Las ventajas
de este último algoritmo es que este da la ruta efectiva
cuando el primero solo da una aproximación. Pero cuenta con la desventaja de que no es un standard
y por tanto es posible que haya routers (gateways) que no lo
implementen. Es por esto que en este trabajo se implementará el primer
algoritmo.
