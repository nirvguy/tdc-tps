\subsection{Universidad de La Habana}

En este experimento analizaremos como se comportan los paquetes enviados hasta la web de la Universidad de La Habana\footnote{http://www.uh.cu/}. Al momento de realizarse el experimento, su IP es 200.55.139.216.

 \includegraphics[scale=0.8]{imagenes/rtts_habana.png}

Podemos observar que ningun cable es considerado transatlantico, utilizando el algoritmo expuesto anteriormente, si bien se puede ver una diferencia considerable de RTT en la IP 195.22.222.15 y 89.221.37.151. Haremos un mejor an\'alisis utilizando geolocalizaci\'on para intentar entender el camino del paquete.\\

Tanto geoiptool como plotip nos dan dos caminos bastante distintos como muestran las tablas a continuaci\'on
 
\begin{table}
\centering
\begin{tabular}{|c|c|}
10.242.1.17 & No puede procesar \\
195.22.222.15 & Italia  \\
89.221.37.151 & Italia \\
200.0.16.25 & Cuba \\
\end{tabular}
\caption{Utilizando geoiptool}
\end{table}

\begin{table}
\centering
\begin{tabular}{|c|c|c|}
10.242.1.17 & Aachen & Alemania \\
195.22.222.15 & Chicago & Estados Unidos  \\
89.221.37.151 &  & Grecia \\
200.0.16.25 & La Habana & Cuba \\
\end{tabular}
\caption{Utilizando plotip}
\end{table}

Primero vemos que en plotip tiene poco sentido el camino de ir de Alemania a Estados Unidos a Grecia para luego volver a Cuba, adem\'as de que no se condice con los RTTs medidos. Al no poder ser procesada la IP 10.242.1.17 por geoiptool nos hace pensar que el caso de Alemania puede estar mal. Con esto en mente, lo mas factible viendo los resultados de plotip es el camino Buenos Aires - Estados Unidos - Cuba. Descartamos Grecia porque creemos que no tiene sentido hacer ese trayecto por la localizaci\'on tan cercana de Cuba a Estados Unidos.\\

En geoiptool, nos da un camino que puede ser valido ya que es ir a Italia para luego ir a Cuba. Si bien la distancia de Buenos Aires a Estados Unidos es tan grande como la que separa a algunos continentes, resulta raro que no haya detectado ningun camino intercontinental nuestro algoritmo. Esto nos hace ver como mas posible entre ambos caminos el primero. Igualmente es posible que el nivel de corte calculado con el metodo de Cimbala sea muy grande en este caso.\\

De esta manera tenemos: PONER FOTO DEL MAPITA.
