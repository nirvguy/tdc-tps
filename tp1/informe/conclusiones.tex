A modo general pudimos observar que en el caso
de la primer fuente, en todas las redes el 
símbolo destacado, o mas probable, es el unicast.
Como ya se explicó previamente esto puede deberse
a la poca cantidad de paquetes de control que suceden
en la red con respecto a la totalidad de paquetes.


Se confirmaron las hipótesis que se presentaron sobre los nodos destacados de
la fuente $S_2$. Es decir, todos estos, tenían algún propósito especial para la
red (router, impresora, access points, etc.) como lo de los niveles
de compresibilidad de los distintos tipos de redes (Cableada, Wifi).


Otra punto importante es qué se observo en todas la redes
menor cantidad de dispositivos en el gráfico de la topología
que IP's en los gráficos de probabilidad e información de la fuente $S_2$.
Esto puede deberse a que haya nodos que tenga varias IP's
o que se enviaron mensajes \textit{who-has} con IP's que no existían o dejaron
de existir en la red.

Para concluir, nos divirtió analizar las distintas redes y poder elaborar hipótesis
y comprobar resultados de los experimentos realizados. De esta forma pudimos iteriorizarnos
y comprender mejor las redes que utilizamos habitualmente.
