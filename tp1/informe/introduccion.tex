A lo largo de la historia las comunicaciones han demostrado ser un factor
indispensable en lo referente al desarrollo y buen funcionamiento de nuestras
sociedades. Sin embargo no fue hasta hace mucho que se desarrollo y formalizó
con rigurosidad una teoría matemática con respecto a ellas. Fue recien Claude
Elwood Shannon quien tras trabajar por muchos años en los laboratorio Bell,
presentó en 1948 su trabajo titulado \textit{A Mathematical Theory of
Communication} (Ver \cite{shannon}). En ella abstrae y define varios conceptos y propiedades comunes
a los distintos medios de transmición que permitieron medir y cuantificar las
cualidades de un canal. Entre ellas introduce el concepto de
\textit{Información} asociada a cada símbolo de una \textit{Fuente}, y junto
con ella la \textit{Entropía}. Mediante el presente trabajo haremos uso de
estos conceptos para estudiar y analizar algunas redes de uso cotidiano.






